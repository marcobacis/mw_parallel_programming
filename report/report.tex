\documentclass[a4paper, 10pt]{article}
\usepackage{comment} % enables the use of multi-line comments (\ifx \fi) 
\usepackage{lipsum} %This package just generates Lorem Ipsum filler text. 
\usepackage{fullpage} % changes the margin
\usepackage{graphicx}
\usepackage{xcolor}
\usepackage[]{algorithm2e}

\begin{document}
%Header-Make sure you update this information!!!!
\noindent
\large\textbf{Middleware Project - MPI/OpenMP} \hfill \textbf{Marco Bacis} 873199 \\
Prof. Guinea \hfill \textbf{Daniele Cattaneo} 874757 \\
Prof. Mottola

\section*{Problem Statement}
The goal of the project is to develop a system to provide analytics over high velocity sensor data originating from a soccer game.
The goal of the analysis is to report and update the ball possession time for each player and team periodically during the game.
The data used comes from the DEBS2013 challenge, in which a number of wireless sensors embedded in the shoes and a ball were used during a soccer match, spanning the whole duration of the game
The main requirements is to use OpenMP and/or MPI to compute the real-time statistics.

\subsection*{Ball possession conditions}
A player is considered in possession of the ball when
\begin{itemize}
\item He is the player closest to the ball
\item He is not farther than K meters from the ball
\item Ball possession is undefined whenever the game was paused
\end{itemize}

\subsection*{Additional Informations}
\begin{itemize}
    \item Arm/shoe attached sensors sends data at 200Hz, while the ball has a 2kHz rate
    \item As found from the DEBS2013 challenge paper, the field is half the dimension of a standard soccer field (66 x 52 meters)
    \item Each player (and even the referee) has 2 sensors, one for each arm. The goalkeeper has 4 sensors (1 more for each arm)
    \item The data from the sensors starts before the game begins and stops after the game ends, including the time between the two halves
    \item There is no sensor data during the last 2.5 minutes before the first half ends
    \item 4 balls are used during the game, but only the one inside the field is used for the possession computation
\end{itemize}

\section*{Implementation}

In order to compute the ball possessions, the algorithm must divide the sensor data into a set of {\bf time-steps} based on the sensor's data rate.
After that, for each timestep the nearest player can be found and its possession time updated.
In this case, given that the slowest sensor has a 200Hz rate, the timestep-based grouping of the records is done by dividing each record's timestamp (in picoseconds) by $1 / 200 = 0.005s = 5\,000\,000\,000ps$.
This is done while loading the game records file.


The ball sensor outputs data at 2kHz, so for each timestep group there are at least 10 ball records (usually more, as many ball sensors are on at the same time, even if just one ball is used).
To compute the ball possession, for each group the nearest player is found for each of the current ball's positions, and the timestep (5ms) is distributed among those distinct positions.
The algorithm's pseudocode is shown in Algorithm~\ref{algorithm:main_loops}.

\begin{algorithm}[tb]
    \SetKwInOut{Input}{input}
    \SetKwInOut{Output}{output}

    \Input{Full game sensor records, game interruptions, K, T}
    \Output{Cumulated possession results every T seconds}
    \BlankLine
    \ForEach{game step $s$ in the game records vector}{
        \If{game is not interrupted at time $s.timestep$}{
            $B \longleftarrow \emptyset$\;
            $P \longleftarrow \emptyset$\;
            \tcp{First loop, to gather just the balls records}
            \ForEach{record $r \in s$} {
                \uIf{$r$ is a ball {\bf and} $r$ is inside the field}{
                    $B \longleftarrow B \cup r$\;
                }
                \uElseIf{$r$ is a player}{
                    $P \longleftarrow P \cup r$\;
                }
            }
            \tcp{Second loop to compute the actual possession}
            $increment \longleftarrow 5ms / B.size$\;
            \ForEach{ball record $b \in B$}{
                $minDist \longleftarrow INT\_MAX$\;
                $minPlayer \longleftarrow \emptyset$\;
                \ForEach{player record $p \in P$}{
                    $dist \longleftarrow p.x^2 + p.y^2$ \tcc*[r]{ground distance}
                    \If{$dist < minDist$}{
                        $minDist \longleftarrow dist$\;
                        $minPlayer \longleftarrow p$\;
                    }
                }
                \If{$minDist < K$}{
                    $possession[minPlayer] \longleftarrow increment$
                }
            }
        }
    }
    \caption{Main algorithm for ball possession computation}
    \label{algorithm:main_loops}
\end{algorithm}


\end{document}
